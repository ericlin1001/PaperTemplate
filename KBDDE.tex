\documentclass[journal]{IEEEtran}
\usepackage{booktabs}
\input{header.tex}

\begin{document}

\title{Keep-Busy Distributed Differential Evolution \\for Power Electronic Circuit Optimization}

\author{
	Jun-Hao~Lin,~\IEEEmembership{Student~Member,~IEEE,}
	Zhi-Hui~Zhan~(Corresponding~Author),~\IEEEmembership{Member,~IEEE,}
	Xiao-Fang~Liu,~\IEEEmembership{Student~Member,~IEEE,}
	and~Jun~Zhang,~\IEEEmembership{Senior~Member,~IEEE}% <-this % stops a space
	\thanks{
		J.-H. Lin, X.-F. Liu, Z.-H. Zhan, and J. Zhang are with the Department of
		Computer Science, Sun Yat-Sen University, Guangzhou, 510275, China,
		with the Key Laboratory of Machine Intelligence and Advanced Computing
		(Sun Yat-sen University), Ministry of Education, China, with the
		Engineering Research Center of Supercomputing Engineering Software (Sun
		Yat-sen University), Ministry of Education, China, and also with the Key
		Laboratory of Software Technology, Education Department of Guangdong
		Province, China. Zhi-Hui Zhan is the corresponding author:
		zhanzhh@mail.sysu.edu.cn; zhanapollo@163.com.
	}%
	\thanks{
		This work was partially supported by the National Natural Science
		Foundations of China (NSFC) with No. 61402545, the Natural Science
		Foundations of Guangdong Province for Distinguished Young Scholars with
		No. 2014A030306038, the Project for Pearl River New Star in Science and
		Technology with No. 201506010047, the NSFC Key Program with No.
		61332002, and the Fundamental Research Funds for the Central Universities
		(15lgzd08).
	}
}

\maketitle



\begin{abstract}
	Differential evolution is a effective algorithm to solve problems over continuous space.
	However, some problems's evluating functions are time-consuming, which leads to the total time to
	solve the problem will exceed the limit. To speed up the solving process, 
	a keep-busy distributed differential evolution algorithm is proposed.
\end{abstract}

\begin{IEEEkeywords}
	Differential evolution(DE), distributed computation, time-consuming problem.
\end{IEEEkeywords}


\section{Introduction}
\IEEEPARstart{T}{his} demo file is intended to serve as a ``starter file''
for IEEE journal papers produced under \LaTeX\ using
IEEEtran.cls version 1.8b and later.
% You must have at least 2 lines in the paragraph with the drop letter
% (should never be an issue)\\

I wish you the best of success.

\subsection{Subsection2 Heading Here}
Subsection text here2.
Subsection text here2.
Subsection text here2.
Subsection text here2.
Subsection text here2.
Subsection text here2.
Subsection text here2.
Subsection text here2.
Subsection text here2.
Subsection text here2.
Subsection text here2.
Subsection text here2.
Subsection text here2.
Subsection text here2.
Subsection text here2.
Subsection text here2.
Subsection text here2.
Subsection text here2.
Subsection text here2.
Subsection text here2.
Subsection text here2.
Subsection text here2.
Subsection text here2.
Subsection text here2.
Subsection text here2.
Subsection text here2.
Subsection text here2.
Subsection text here2.
Subsection text here2.
Subsection text here2.
Subsection text here2.
Subsection text here2.
Subsection text here2.
Subsection text here2.
Subsection text here2.
Subsection text here2.
Subsection text here2.
Subsection text here2.
Subsection text here2.
Subsection text here2.
Subsection text here2.
Subsection text here2.
Subsection text here2.
Subsection text here2.
Subsection text here2.
Subsection text here2.
Subsection text here2.
Subsection text here2.
Subsection text here2.
Subsection text here2.
Subsection text here2.
Subsection text here2.
Subsection text here2.
Subsection text here2.
Subsection text here2.
Subsection text here2.
Subsection text here2.
Subsection text here2.
Subsection text here2.
Subsection text here2.
Subsection text here2.
Subsection text here2.
Subsection text here2.
Subsection text here2.


\subsubsection{Subsubsection3 Heading Here}
Subsubsection text here3.
Subsubsection text here3.
Subsubsection text here3.
Subsubsection text here3.
Subsubsection text here3.
Subsubsection text here3.
Subsubsection text here3.
Subsubsection text here3.
Subsubsection text here3.
Subsubsection text here3.
Subsubsection text here3.
Subsubsection text here3.
Subsubsection text here3.
Subsubsection text here3.
Subsubsection text here3.
Subsubsection text here3.
Subsubsection text here3.
Subsubsection text here3.
Subsubsection text here3.
Subsubsection text here3.
Subsubsection text here3.
Subsubsection text here3.
Subsubsection text here3.
Subsubsection text here3.
Subsubsection text here3.
Subsubsection text here3.
Subsubsection text here3.
Subsubsection text here3.
Subsubsection text here3.
Subsubsection text here3.
Subsubsection text here3.
Subsubsection text here3.
Subsubsection text here3.
Subsubsection text here3.
Subsubsection text here3.
Subsubsection text here3.
Subsubsection text here3.
Subsubsection text here3.\\
Subsubsection text here3.
Subsubsection text here3.
Subsubsection text here3.
Subsubsection text here3.
Subsubsection text here3.
Subsubsection text here3.
Subsubsection text here3.
Subsubsection text here3.
Subsubsection text here3.
Subsubsection text here3.
Subsubsection text here3.
Subsubsection text here3.
Subsubsection text here3.\\
Subsubsection text here3.
Subsubsection text here3.
Subsubsection text here3.
Subsubsection text here3.
Subsubsection text here3.
Subsubsection text here3.
Subsubsection text here3.
Subsubsection text here3.
Subsubsection text here3.
Subsubsection text here3.
Subsubsection text here3.
Subsubsection text here3.
Subsubsection text here3.
Subsubsection text here3.
Subsubsection text here3.
Subsubsection text here3.
Subsubsection text here3.
Subsubsection text here3.


% put all your \label just after \caption rather than within \caption{}.
%
% Reminder: the "draftcls" or "draftclsnofoot", not "draft", class
% option should be used if it is desired that the figures are to be
% displayed while in draft mode.
%
\begin{figure}[!t]
\centering
\includegraphics[width=2.5in]{myfigure.jpg}
\caption{Simulation results for the network.}
\label{fig_sim}
\end{figure}

% Note that the IEEE typically puts floats only at the top, even when this
% results in a large percentage of a column being occupied by floats.


% An example of a double column floating figure using two subfigures.
% (The subfig.sty package must be loaded for this to work.)
% The subfigure \label commands are set within each subfloat command,
% and the \label for the overall figure must come after \caption.
% \hfil is used as a separator to get equal spacing.
% Watch out that the combined width of all the subfigures on a 
% line do not exceed the text width or a line break will occur.
%
\begin{figure*}[!t]
\centering
\subfloat[Case I]{\includegraphics[width=2.5in]{myfigure}%
\label{fig_first_case}}
\hfil
\subfloat[Case II]{\includegraphics[width=2.5in]{myfigure}%
\label{fig_second_case}}
\caption{Simulation results for the network.}
\label{fig_sim2}
\end{figure*}
\newpage
I'm refering to Fig.\ref{fig_first_case}...

%  If a subcaption is not desired, just leave its contents blank, e.g., \subfloat[].

% last word of the caption. Table text will default to \footnotesize as the IEEE normally uses this smaller font for tables.
% The \label must come after \caption as always.

\begin{table}[!t]
% increase table row spacing, adjust to taste
\renewcommand{\arraystretch}{1.3}
% if using array.sty, it might be a good idea to tweak the value of \extrarowheight as needed to properly center the text within the cells
\caption{An Example of a Table}
\label{table_example}
\centering
% Some packages, such as MDW tools, offer better commands for making tables
	\begin{tabular}{cc}
		\toprule
		One & Two\\
		\midrule
		Three & Four\\
		Three & Four\\
		\bottomrule
	\end{tabular}
\end{table}


% Note that the IEEE does not put floats in the very first column
% - or typically anywhere on the first page for that matter. Also,
% in-text middle ("here") positioning is typically not used, but it
% is allowed and encouraged for Computer Society conferences (but
% not Computer Society journals). Most IEEE journals/conferences use
% top floats exclusively. 
% Note that, LaTeX2e, unlike IEEE journals/conferences, places
% footnotes above bottom floats. This can be corrected via the
% \fnbelowfloat command of the stfloats package.




\section{Conclusion}
The conclusion goes here.






% Can use something like this to put references on a page
% by themselves when using endfloat and the captionsoff option.
\ifCLASSOPTIONcaptionsoff
\newpage
\fi

% trigger a \newpage just before the given reference
% number - used to balance the columns on the last page
% adjust value as needed - may need to be readjusted if
% the document is modified later
%\IEEEtriggeratref{8}
% The "triggered" command can be changed if desired:
%\IEEEtriggercmd{\enlargethispage{-5in}}

% references section

% can use a bibliography generated by BibTeX as a .bbl file
% BibTeX documentation can be easily obtained at:
% http://mirror.ctan.org/biblio/bibtex/contrib/doc/
% The IEEEtran BibTeX style support page is at:
% http://www.michaelshell.org/tex/ieeetran/bibtex/
%\bibliographystyle{IEEEtran}
% argument is your BibTeX string definitions and bibliography database(s)
%\bibliography{IEEEabrv,../bib/paper}
%
% <OR> manually copy in the resultant .bbl file
% set second argument of \begin to the number of references
% (used to reserve space for the reference number labels box)
\begin{thebibliography}{1}

	\bibitem{IEEEhowto:kopka}
		H.~Kopka and P.~W. Daly, \emph{A Guide to \LaTeX}, 3rd~ed.\hskip 1em plus
		0.5em minus 0.4em\relax Harlow, England: Addison-Wesley, 1999.

\end{thebibliography}

% biography section
% 
% If you have an EPS/PDF photo (graphicx package needed) extra braces are
% needed around the contents of the optional argument to biography to prevent
% the LaTeX parser from getting confused when it sees the complicated
% \includegraphics command within an optional argument. (You could create
% your own custom macro containing the \includegraphics command to make things
% simpler here.)
%\begin{IEEEbiography}[{\includegraphics[width=1in,height=1.25in,clip,keepaspectratio]{mshell}}]{Michael Shell}
% or if you just want to reserve a space for a photo:

%\vfill
\begin{IEEEbiography}[{\includegraphics[width=1in,height=1.25in,clip,keepaspectratio]{junhao}}]{Jun-Hao Lin}
afadfadf
\end{IEEEbiography}

\begin{IEEEbiographynophoto}{Zhan-Zhi Hui}
	Biography text here.
\end{IEEEbiographynophoto}

% insert where needed to balance the two columns on the last page with
% biographies
%\newpage

% if you will not have a photo at all:
\begin{IEEEbiographynophoto}{Xiao-Fang Liu}
	Biography text here.
\end{IEEEbiographynophoto}

%\vfill
% Can be used to pull up biographies so that the bottom of the last one
% is flush with the other column.
%\enlargethispage{-5in}



% that's all folks
\end{document}
